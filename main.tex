% !TeX spellcheck = en_US
%!TEX encoding = UTF-8 Unicoded
\special{dvipdfmx:config z 0}
\documentclass[10pt,aspectratio=169]{beamer} % 169 为4:3比例
\setbeamercovered{transparent=20}
\usetheme[
%  showheader,
%  red,
%  purple,
%   gray,
lightBlue,
%  colorblocks,
%  noframetitlerule,
]{chd}
\usepackage[T1]{fontenc}
\usepackage[utf8]{inputenc}
\usepackage{lipsum}
\providecommand{\lishu}{\CJKfamily{song}}
\usepackage{xeCJK}
\usepackage{listings}
\usefonttheme{professionalfonts}
\def\mathfamilydefault{\rmdefault}
\usepackage{amsmath}
\usepackage{multirow}
\usepackage{booktabs}
% 以下设置表格内容位置格式
\usepackage{bm}
\usepackage{array}
% 以下设置插入图片及图片编号
\setbeamertemplate{caption}[numbered]
\usepackage{subfigure}
\usepackage{graphicx}
\usepackage{float} 
\usepackage{chngcntr}
\counterwithin{figure}{section}
\counterwithin{table}{section}
\usepackage{caption}
\captionsetup{figurename=图}
\captionsetup{tablename=表}
% 以下设置首行缩进2字符
\usepackage{indentfirst} 
\setlength{\parindent}{2em} 

% beamer中对文献进行引用
\usepackage{cite}
\usepackage{natbib}
\usepackage{gbt7714}
\bibliographystyle{gbt7714-numerical}

% 以下设置目录格式
\setbeamertemplate{section in toc}{\hspace*{1em}\inserttocsectionnumber.~\inserttocsection\par}
\setbeamertemplate{subsection in toc}{\hspace*{2em}\inserttocsectionnumber.\inserttocsubsectionnumber.~\inserttocsubsection\par}
\setbeamerfont{subsection in toc}{size=\small} % 设置目录样式

%=======================================================================
%===================================================================
%在章节开始前放映章节目录
\AtBeginSection[]{
	\begin{frame}<beamer>
		\frametitle{\quad}
		\textbf{\tableofcontents[currentsection]} 
	\end{frame}
}
%===================================================================
% 在子章节开始前放映子章节目录
%\AtBeginSubsection[]{%
	%	\begin{frame}%
		%		\frametitle{子章节位置}%
		%		\textbf{\tableofcontents[currentsection, currentsubsection]} %
		%	\end{frame}%
	%}
%===================================================================
%===================================================================
\title{基于CA-Markov的土地利用变化预测}
\subtitle{——以陕西省商洛市为例}   % 子标题
\author[]{汇报人\quad 朱家政\\}
%\mail{\emph zhujiazheng@chd.edu.cn}
\institute[Chang'an University]{\large 地球科学与资源学院 \\School of Geosciences \& Resources}
\date{\today}
\titlegraphic[width=3cm]{logo/chdlogobluewhite}{}  % 在首页显示校徽
%===================================================================
%===================================================================
\begin{document}
	
	\maketitle
	\begin{frame}
		\frametitle{目录}
		\tableofcontents
	\end{frame}
%===================================================================
\section{引言}
		\begin{frame}[c]{\secname}
		test\cite{sangSimulationLandUse2011}。
		\end{frame}
%===================================================================
	\section{研究背景}
	\subsection{研究区概况}
		\begin{frame}[c]{\secname}{\subsecname}
		\begin{columns}
			\column{.6\textwidth}
			\qquad 
			\par\qquad  $℃$如图\ref{Fig.main.1}所示。
			\column{.4\textwidth}
			\begin{figure}
				\includegraphics[width=\textwidth]{images/shangluoshi.pdf} 
				\caption{陕西省商洛市位置}
				\label{Fig.main.1} %用于文内引用的标签
			\end{figure}
		\end{columns}	
	\end{frame}
	\subsection{数据来源}	
	\begin{frame}[c]{\secname}{\subsecname}
	\begin{columns}
	\column{.4\textwidth}
	\qquad 级类型,分别是:耕地、林地、草地、灌木地、湿地、水体、苔原、人造地表、裸地、冰川和永久积雪。分类系统定义见表\ref{Table.main.1}。
	\par
	\column{.6\textwidth}
\begin{table}[htbp]
	\tiny
	\centering
	\caption{GlobeLand30分类系统}
	    \begin{tabular}{c| m{24.78em}| c}
	    	\hline
	    	\textcolor[rgb]{ .2,  .2,  .2}{\textbf{类型}} &
	    	\centering\textcolor[rgb]{ .2,  .2,  .2}{\textbf{内容}} &
	    	\textcolor[rgb]{ .2,  .2,  .2}{\textbf{代码}}
	    	\\
	    	\hline
	    	\textcolor[rgb]{ .2,  .2,  .2}{耕地} &
	    	\textcolor[rgb]{ .2,  .2,  .2}{用于种植农作物的土地,包括水田、灌溉旱地、雨养旱地、菜地、牧草种植地、大棚用地、以种植农作物为主间有果树及其他经济乔木的土地,以及茶园、咖啡园等灌木类经济作物种植地。} &
	    	\textcolor[rgb]{ .2,  .2,  .2}{1}
	    	\\
	    	\hline
	    	\textcolor[rgb]{ .2,  .2,  .2}{林地} &
	    	\textcolor[rgb]{ .2,  .2,  .2}{乔木覆盖且树冠盖度超过30\%的土地,包括落叶阔叶林、常绿阔叶林、落叶针叶林、常绿针叶林、混交林,以及树冠盖度为10-30\%的疏林地。} &
	    	\textcolor[rgb]{ .2,  .2,  .2}{2}
	    	\\
	    	\hline
	    	\textcolor[rgb]{ .2,  .2,  .2}{草地} &
	    	\textcolor[rgb]{ .2,  .2,  .2}{天然草本植被覆盖,且盖度大于10\%的土地,包括草原、草甸、稀树草原、荒漠草原,以及城市人工草地等。} &
	    	\textcolor[rgb]{ .2,  .2,  .2}{3}
	    	\\
	    	\hline
	    	\textcolor[rgb]{ .2,  .2,  .2}{灌木地} &
	    	\textcolor[rgb]{ .2,  .2,  .2}{灌木覆盖且灌丛覆盖度高于30\%的土地,包括山地灌丛、落叶和常绿灌丛,以及荒漠地区覆盖度高于10\%的荒漠灌丛。} &
	    	\textcolor[rgb]{ .2,  .2,  .2}{4}
	    	\\
	    	\hline
	    	\textcolor[rgb]{ .2,  .2,  .2}{湿地} &
	    	\textcolor[rgb]{ .2,  .2,  .2}{位于陆地和水域的交界带,有浅层积水或土壤过湿的土地,多生长有沼生或湿生植物。包括内陆沼泽、湖泊沼泽、河流洪泛湿地、森林/灌木湿地、泥炭沼泽、红树林、盐沼等。} &
	    	\textcolor[rgb]{ .2,  .2,  .2}{5}
	    	\\
	    	\hline
	    	\textcolor[rgb]{ .2,  .2,  .2}{水体} &
	    	\textcolor[rgb]{ .2,  .2,  .2}{陆地范围液态水覆盖的区域,包括江河、湖泊、水库、坑塘等。} &
	    	\textcolor[rgb]{ .2,  .2,  .2}{6}
	    	\\
	    	\hline
	    	\textcolor[rgb]{ .2,  .2,  .2}{苔原} &
	    	\textcolor[rgb]{ .2,  .2,  .2}{寒带及高山环境下由地衣、苔藓、多年生耐寒草本和灌木植被覆盖的土地,包括灌丛苔原、禾本苔原、湿苔原、高寒苔原、裸地苔原等。} &
	    	\textcolor[rgb]{ .2,  .2,  .2}{7}
	    	\\
	    	\hline
	    	\textcolor[rgb]{ .2,  .2,  .2}{人造地表} &
	    	\textcolor[rgb]{ .2,  .2,  .2}{由人工建造活动形成的地表,包括城镇等各类居民地、工矿、交通设施等,不包括建设用地内部连片绿地和水体。} &
	    	\textcolor[rgb]{ .2,  .2,  .2}{8}
	    	\\
	    	\hline
	    	\textcolor[rgb]{ .2,  .2,  .2}{裸地} &
	    	\textcolor[rgb]{ .2,  .2,  .2}{植被覆盖度低于10\%的自然覆盖土地,包括荒漠、沙地、砾石地、裸岩、盐碱地等。} &
	    	\textcolor[rgb]{ .2,  .2,  .2}{9}
	    	\\
	    	\hline
	    	\textcolor[rgb]{ .2,  .2,  .2}{冰川和永久积雪} &
	    	\textcolor[rgb]{ .2,  .2,  .2}{由永久积雪、冰川和冰盖覆盖的土地,包括高山地区永久积雪、冰川,以及极地冰盖等。} &
	    	\textcolor[rgb]{ .2,  .2,  .2}{10}
	    	\\
	    	\hline
	    \end{tabular}%
	\label{Table.main.1}%
\end{table}%
	\end{columns}	
	\end{frame}

	\begin{frame}[c]{\secname}{\subsecname}
		GlobeLand30数据采用WGS-84坐标系。其南纬85$°$-北纬85$°$之间的区域投影方式采用UTM投影, 6度分带。对应的陕西省商洛地区数据位于其N49\_30图幅中,如图\ref{Fig.main.2}所示。
			\begin{figure}[htbp] %H为当前位置,!htb为忽略美学标准,htbp为浮动图形
			\centering %图片居中
			\includegraphics[width=0.5\textwidth]{images/n49e30.png} 
			\caption{GlobeLand30数据图幅示意} %最终文档中希望显示的图片标题
			\label{Fig.main.2} %用于文内引用的标签
		\end{figure}
	\end{frame}
	%==================================================
	%==================================================
	\section{研究方法}
	\subsection{Markov模型}
	\begin{frame}[c]{\secname}{\subsecname}
		Markov模型是基于Markov过程理论形成的用来预测事件发生概率的一种方法。使用的前提条件是事件过程具有无后效应特征,即过程在$t_0$时刻所处的状态为已知条件,$t>t_0$时的状态与在在$t_0$\textbf{之前}所处状态无关。一定区域内,不同土地利用类型之间具有相互可转化性\cite{muller1994markov,何春阳2004中国北方未来土地利用变化情景模拟},因此在一定条件下,土地利用的动态演变具有Markov过程的性质。
		\par
		其中,土地利用类型对应Markov过程中的“可能状态”,土地利用类型相互转化的面积数量或比例对应状态转移概率(转移矩阵),公式\ref{Equa.main.1}中的$ S_{t}$即为$t$时刻状态。
		\begin{equation}\label{Equa.main.1}
			S_{t+1}=P_{i j} \times S_{t}
		\end{equation}
		\par
		$P_{i j} $即为转移矩阵,见公式\ref{Equa.main.2}。
		\begin{equation}\label{Equa.main.2}
		P_{i j}=\left[\begin{array}{cccc}
			P_{11} & P_{12} & \cdots & P_{1 n} \\
			P_{21} & P_{22} & \cdots & P_{2 n} \\
			\vdots & \vdots & & \vdots \\
			P_{n 1} & P_{n 2} & \cdots & P_{n n}
		\end{array}\right]
		\end{equation}
		\par
		式中$0<=P_{ij}<1$,且$\sum_{j=1}^{n} P_{i j}=1(i, j=1,2, \cdots, n)$,即每个当前状态的土地类型向下个状态的土地类型转移的概率和为1。
		
	\end{frame}

%=====================================
	\subsection{元胞自动机}
	\begin{frame}[c]{\secname}{\subsecname}
	元胞自动机(Cellular Automata,CA)是一种时间、空间、状态都离散,空间相互作用和时间因果关系都为局部的网格动力学模型,具有模拟复杂系统时空演化过程的能力\cite{孙战利1999空间复杂性与地理元胞自动机模拟研究}。
	\par
	在土地利用变化的模拟中,将研究区划分为元胞,每个元胞都对应着一定的土地利用类型,土地利用变化的历史趋势 、土地适宜性以及相关的政策 、经济因素构成规则,它们共同决定着每个元胞土地利用类型转换的可能性,当可能性超过控制阈值的时候,土地利用类型将发生转化\cite{熊利亚2005基于地理元胞自动机的土地利用变化研究}。
	\end{frame}
%==================================================
	\subsection{Terrset软件}
\begin{frame}[c]{\secname}{\subsecname}
	TerrSet\cite{TerrSetGeospatialMonitoring}是一个综合地理空间软件系统,用于监测和模拟地理空间变化以促进可持续发展。TerrSet软件结合了IDRISI地理信息系统分析和IDRISI图像处理工具等应用程序。本研究使用了Terrset中的IDRISI GIS Analysis——Change/Times Series——MARKOV和CA\_MARKOV工具。
	\begin{figure}
		\centering
		\includegraphics[height=0.6\textheight]{images/jietu}
		\caption{Terrset软件截图}
		\label{Fig.main.3}
	\end{figure}
	
\end{frame}
%==================================================
	\subsection{CA-Markov模型}
	\begin{frame}[c]{\secname}{\subsecname}
	CA-Markov模型结合了元胞自动机和马尔科夫链模型,将连续的空间数据和空间转换信息加入到马尔科夫链分析中,综合了元胞自动机模拟复杂空间变化的能力和Markov模型定量化预测的优势,提高了土地利用类型转化的精度\cite{sangSimulationLandUse2011,hamad2018predicting,subedi2013application}。	
	\begin{table}[htbp]
		\centering
		\begin{tabular}{ccccccccc}
			\hline
			\multicolumn{1}{l}{土地利用类型} &
			\multicolumn{1}{l}{耕地} &
			\multicolumn{1}{l}{林地} &
			\multicolumn{1}{l}{草地} &
			\multicolumn{1}{l}{灌木地} &
			\multicolumn{1}{l}{湿地} &
			\multicolumn{1}{l}{水体} &
			\multicolumn{1}{l}{苔原} &
			\multicolumn{1}{l}{人造地表}
			\\
			\hline
			耕地 &
			0.7080  &
			0.1760  &
			0.0326  &
			0.0000  &
			0.0002  &
			0.0010  &
			0.0000  &
			0.0822 
			\\
			林地 &
			0.0696  &
			0.8068  &
			0.1158  &
			0.0000  &
			0.0001  &
			0.0016  &
			0.0000  &
			0.0061 
			\\
			草地 &
			0.0330  &
			0.3782  &
			0.5723  &
			0.0000  &
			0.0003  &
			0.0010  &
			0.0000  &
			0.0151 
			\\
			灌木地 &
			0.1429  &
			0.1429  &
			0.1429  &
			0.0000  &
			0.1429  &
			0.1429  &
			0.1429  &
			0.1429 
			\\
			湿地 &
			0.0298  &
			0.0178  &
			0.0154  &
			0.0000  &
			0.8087  &
			0.1188  &
			0.0000  &
			0.0096 
			\\
			水体 &
			0.0223  &
			0.2269  &
			0.2864  &
			0.0000  &
			0.0902  &
			0.3351  &
			0.0000  &
			0.0392 
			\\
			苔原 &
			0.1429  &
			0.1429  &
			0.1429  &
			0.1429  &
			0.1429  &
			0.1429  &
			0.0000  &
			0.1429 
			\\
			人造地表 &
			0.1708  &
			0.0359  &
			0.0261  &
			0.0000  &
			0.0001  &
			0.0015  &
			0.0000  &
			0.7657 
			\\
			\hline
		\end{tabular}%
	\label{Table.2}
		\caption{Markov变化概率矩阵}
	\end{table}%
\qquad 在Terrset软件中,首先使用2000、2010两期土地利用分类数据生成Markov变化矩阵,参数为默认设置(5*5的元胞空间;迭代10次)结果如表3.1所示。在Terrset中设置间隔单位为10年,因此得出的概率解释为每年每个类别向其他类别转换的概率。
	\end{frame}
		\begin{frame}[c]{\secname}{\subsecname}
		使用表3.1变化矩阵的Morkov模型,利用2020年的土地利用分类数据对2030年土地利用变化做预测,依然为默认参数,结果如图\ref{Fig.3}。
		\begin{figure}[htbp]
			\centering
			\subfigure[2020年土地利用分类数据]{
				\begin{minipage}[t]{0.45\linewidth}
					\centering
					\includegraphics[width=\linewidth]{images/2020c}
				\end{minipage}%
			}%
			\subfigure[2030年预测结果]{
				\begin{minipage}[t]{0.45\linewidth}
					\centering
					\includegraphics[width=\linewidth]{images/2030c}
				\end{minipage}%
			}%
			\caption{2020年土地利用分类数据与2030年预测结果对比}
			\label{Fig.3}
		\end{figure}
	
	\end{frame}
%====================================================================
%==================================================================== 	
	\section{研究结果与讨论}
%====================================================================
	\subsection{研究结果}
	\begin{frame}[c]{\secname}{\subsecname}
		使用CA-Markov模型预测的2030年土地利用变化结果见表\ref{Table.3}。
		% Table generated by Excel2LaTeX from sheet 'Sheet1'
		\begin{table}[htbp]
			\centering
			\caption{2030年土地利用变化}
			\begin{tabular}{cccccc}
				\toprule
				\multirow{2}[4]{*}{\textbf{土地利用类型}} &
				\multicolumn{2}{c}{\textbf{2020年}} |&
				\multicolumn{2}{c}{\textbf{2030年}} &
				\multirow{2}[4]{*}{\textbf{年均变化率}}
				\\
				\cmidrule{2-5}     &
				\textbf{面积/平方公里} &
				\textbf{比例} &
				\textbf{面积/平方公里} &
				\textbf{比例} &
				
				\\
				\midrule
				\textbf{耕地} &
				2514.31 &
				12.99\% &
				2919.76 &
				15.09\% &
				0.21\%
				\\
				\textbf{林地} &
				14932.30 &
				77.15\% &
				13187.64 &
				68.14\% &
				-0.90\%
				\\
				\textbf{草地} &
				1622.01 &
				8.38\% &
				2750.16 &
				14.21\% &
				0.58\%
				\\
				\textbf{湿地} &
				15.01 &
				0.08\% &
				15.48 &
				0.08\% &
				0.00\%
				\\
				\textbf{水体} &
				15.21 &
				0.08\% &
				11.24 &
				0.06\% &
				0.00\%
				\\
				\textbf{人造地表} &
				255.60 &
				1.32\% &
				470.15 &
				2.43\% &
				0.11\%
				\\
				\bottomrule
			\end{tabular}%
			\label{Table.3}%
		\end{table}%
		其中林地面积呈减少趋势,草地、耕地、人造地表面积逐年增加。		
	\end{frame}
	%=======================================================================
	\subsection{模型验证}
	\begin{frame}[c]{\secname}{\subsecname}
		为了确保模拟结果的可靠性,我们需要对模型进行验证。使用2000、2010两期土地利用分类数据生成的Markov变化矩阵,利用2010年土地利用分类数据预测2020年土地利用数据。通过统计预测的2020年各类用地预测面积与来自GlobeLand30的实际面积进行比较,并计算总精度$
		E=\sum{\left[ \left| \left( m_{i0}-m_{it} \right) \right|/m_{i0} \right]}\times 100\%
		$。验证结果见表\ref{Table.4}。
	% Table generated by Excel2LaTeX from sheet 'Sheet3'
	\begin{table}[htbp]
		\centering
		\caption{基于2020年预测结果与实际分类结果比对的模型验证}
		\begin{tabular}{crrrc}
			\toprule
			\textbf{土地利用类型} &
			\multicolumn{1}{c}{\textbf{实际面积}} &
			\multicolumn{1}{c}{\textbf{模型预测面积}} &
			\multicolumn{1}{c}{\textbf{面积误差}} &
			\textbf{总精度}
			\\
			\midrule
			\textbf{耕地} &
			2514.31 &
			3534.93 &
			40.59\% &
			\multirow{6}[2]{*}{\textbf{89.65\%}}
			\\
			\textbf{林地} &
			14932.30 &
			15198.52 &
			1.78\% &
			
			\\
			\textbf{草地} &
			1622.01 &
			2180.29 &
			34.42\% &
			
			\\
			\textbf{湿地} &
			15.01 &
			13.50 &
			-10.07\% &
			
			\\
			\textbf{水体} &
			15.21 &
			15.81 &
			3.98\% &
			
			\\
			\textbf{人造地表} &
			255.60 &
			411.37 &
			60.94\% &
			
			\\
			\bottomrule
		\end{tabular}%
		\label{Table.4}%
	\end{table}%
	
	\end{frame}

	%====================================================================
	%====================================================================	
	\subsection{不足}
	\begin{frame}{\secname}{\subsecname}
		\large\begin{itemize}
			\item 未使用IOU检验预测结果的空间精度;
			\item 利用同一模型对不同基期进行预测和验证;
			\item 生成模型的参数及预测的参数均为默认,未经过验证;
			\item 未阐明研究现实意义;
			\item 模型中缺少环境驱动力因素如高程、人口、经济等。
			
		\end{itemize}
	\end{frame}
	
	%====================================================================
	%====================================================================
	\section{参考文献}
	\begin{frame}[allowframebreaks]{\secname}
			
			\tiny\bibliography{ref.bib}
	\end{frame}
	%====================================================================
	%====================================================================
	% Thank you page
	\beamertemplateshadingbackground{structure.fg!30}{structure.fg} % 两层设置渐变色
	\begin{frame}%[plain]
		\vfill
		\centering
		{
			\centering \Huge \color{white} Thanks! \\谢谢!
		}
		\vfill
	\end{frame}
	%===============================================================
\end{document}
