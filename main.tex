% !TeX spellcheck = en_US
%!TEX encoding = UTF-8 Unicoded
\special{dvipdfmx:config z 0}
\documentclass[10pt,aspectratio=169]{beamer} % 169 为4:3比例
\setbeamercovered{transparent=20}
\usetheme[
%  showheader,
%  red,
%  purple,
%   gray,
lightBlue,
%  colorblocks,
%  noframetitlerule,
]{chd}
\usepackage[T1]{fontenc}
\usepackage[utf8]{inputenc}
\usepackage{lipsum}
\providecommand{\lishu}{\CJKfamily{song}}
\usepackage{xeCJK}
\usepackage{listings}
\usefonttheme{professionalfonts}
\def\mathfamilydefault{\rmdefault}
\usepackage{amsmath}
\usepackage{multirow}
\usepackage{booktabs}
% 以下设置表格内容位置格式
\usepackage{bm}
\usepackage{array}
% 以下设置插入图片及图片编号
\setbeamertemplate{caption}[numbered]
\usepackage{subfigure}
\usepackage{graphicx}
\usepackage{float} 
\usepackage{chngcntr}
\counterwithin{figure}{section}
\counterwithin{table}{section}
\usepackage{caption}
\captionsetup{figurename=图}
\captionsetup{tablename=表}
% 以下设置首行缩进2字符
\usepackage{indentfirst} 
\setlength{\parindent}{2em} 

% beamer中对文献进行引用
\usepackage{cite}
\usepackage{natbib}
\usepackage{gbt7714}
\bibliographystyle{gbt7714-numerical}

% 以下设置目录格式
\setbeamertemplate{section in toc}{\hspace*{1em}\inserttocsectionnumber.~\inserttocsection\par}
\setbeamertemplate{subsection in toc}{\hspace*{2em}\inserttocsectionnumber.\inserttocsubsectionnumber.~\inserttocsubsection\par}
\setbeamerfont{subsection in toc}{size=\small} % 设置目录样式

%=======================================================================
%===================================================================
%在章节开始前放映章节目录
\AtBeginSection[]{
	\begin{frame}<beamer>
		\frametitle{\quad}
		\textbf{\tableofcontents[currentsection]} 
	\end{frame}
}
%===================================================================
% 在子章节开始前放映子章节目录
%\AtBeginSubsection[]{%
	%	\begin{frame}%
		%		\frametitle{子章节位置}%
		%		\textbf{\tableofcontents[currentsection, currentsubsection]} %
		%	\end{frame}%
	%}
%===================================================================
%===================================================================
\title{基于Swin-Transformer的遥感影像水体提取}
\subtitle{——以哨兵2号影像某区域为例}   % 子标题
\author[JiaZheng Zhu]{朱家政}
\mail{\emph zhujiazheng@chd.edu.cn}
\institute[Chang'an University]{\large 地球科学与资源学院 \\School of Geosciences \& Resources}
\date{\today}
\titlegraphic[width=3cm]{logo/chdlogobluewhite}{}  % 在首页显示校徽
%===================================================================
%===================================================================
\begin{document}
	
	\maketitle
	
	%===================================================================
	%============================FRAME===========================
	%===================================================================
	%===================================================================
	\section{引言}
		\begin{frame}[c]{\secname}


	\end{frame}
	%==================================================
	\section{研究背景}
	\subsection{研究区概况}
		\begin{frame}[c]{\secname}{\subsecname}
	
	\end{frame}
	%=====================================

	%=======================================================================
	%=======================================================================
	
	
	%=======================================================================
	%=======================================================================
	% Thank you page
	\beamertemplateshadingbackground{structure.fg!30}{structure.fg} % 两层设置渐变色
		\begin{frame}%[plain]
		\vfill
		\centering
		{
			\centering \Huge \color{white} Thanks! \\谢谢!
		}
		\vfill
	\end{frame}
	%===============================================================
\end{document}
